\documentclass[twocolumn]{article}
\usepackage{graphics}
\usepackage{graphicx}
\usepackage[utf8]{inputenc}
\usepackage{hyperref}
\usepackage{natbib}
\usepackage{graphicx}
\usepackage{svg}
\usepackage{pdfpages}
%%\usepackage{floatrow}

\graphicspath{ {./assets/images/} }

%links
\hypersetup{
    colorlinks=true,
    linkcolor=blue,
    filecolor=magenta,      
    urlcolor=cyan,
}

%Pseudocode
\usepackage[german,vlined,longend,ruled,linesnumbered]{algorithm2e}
\SetKw{KwDownTo}{downto}
\SetKw{KwAnd}{and}
\SetKw{KwOr}{or}
\SetKwBlock{DoParallel}{do in parallel}{end}
\usepackage{xcolor}
\newcommand\mycommfont[1]{\footnotesize\ttfamily\textcolor{blue}{#1}}
\SetCommentSty{mycommfont}
\DontPrintSemicolon

\newcommand{\aetitle}{Efficiently updating Customizable Contraction Hierarchies} % Title of the 

\newcommand{\studentOne}{Marius Hahn} % Name 2

\begin{document}

\twocolumn[{\begin{small}
                \begin{minipage}{0.5 \linewidth}
                    Seminar Summer Term 2022
                \end{minipage}
                \begin{minipage}{0.5\linewidth}
                    \begin{flushright}
                        \studentOne
                    \end{flushright}
                \end{minipage}
            \end{small}}
        {\begin{center}
                \begin{sffamily}\Large\bfseries \aetitle \end{sffamily}
            \end{center}}
    \vskip 3em]

\section{Intro}

This is a seminar on "Efficient shortest path index maintenance on dynamic
road networks with theoretical guarantees" \cite{Ouyang2020}. The paper is
about finding the shortest path in road networks. By \textit{shortest} it is meant,
the path that requires the less time to get from source $s$ to a target $t$. The route
network is modeled as a directed graph $G(V,E)$ where each street crossing represents a
vertex $v \epsilon V$ and each road between crossings represents an edge $e \epsilon E$.
The most basic and solid method to find shortest paths between vertices in a graph is
Dijkstra's algorithm \cite{Dijkstra1959}. This algorithm is proofed to always return the
correct shortest path but it is not fast enough for just in time route planning on large
road networks as we know it from services like Google Maps.
\\
There are many
different approaches that try to speed up shortest path queries by precomputing
any different kind of index structure before doing the shortest path query. The index
structure discussed in "Efficient shortest path index maintenance on dynamic road networks with theoretical guarantees"
\cite{Ouyang2020} is CH (Contraction Hierarchies)\cite{Geisberger2012}
with some extension. This extension is CCH (Customizable Contraction Hierarchies)
\cite{Dibbelt2014}. Although the authors of "Efficient shortest path index maintenance on dynamic road networks with theoretical guarantees"
never mention the term
CCH their approaches builds the same index structure. This is a pity, as for this part
they kind of reinvent the wheel. 
\\
The difference lies in updating
the CCH index structure. For updating the CCH, the authors of "Efficient shortest path index maintenance on dynamic road networks with theoretical guarantees"
\cite{Ouyang2020} will use yet another
index structure called \textit{SS-Graph} that helps to exactly identify the shortcuts that
have to be updated, after some edge weight has changed. 
\\
This works purpose is to show that the update procedure using the \textit{SS-Graph} described in 
"Efficient shortest path index maintenance on dynamic road networks with theoretical guarantees" \cite{Ouyang2020} can be taken as an 
extension to CCH. 
\\
In todays implementation, the whole
index structure is recomputed periodically. This is an valid approach as it is fast enough
to stay accurate for route planning in road networks.
\\
In "Efficient shortest path index maintenance on dynamic road networks with theoretical guarantees" \cite{Ouyang2020} the authors would like to find a way that make it possible to handle streaming
updates. Therefore it is necessary to handle single weight decreases
and increases. To do so another index structure the \textit{SS-Graph} is introduced.
This index structure is a helper structure to identify the shortcuts
that have to be updated in the CCH.
\\
The disadvantage of this \textit{SS-Graph} is that the overall space consumption rises.
This can be a deal breaker for large networks.
Finally they introduce a way to create the necessary \textit{SS-Graph} on the fly. Which
is only a part of the whole structure.
Sadly the exact way, how, is missing in the paper "Efficient shortest path index maintenance on dynamic road networks with theoretical guarantees" \cite{Ouyang2020}.



\bibliographystyle{plain}
\bibliography{assets/references} 




\end{document}
